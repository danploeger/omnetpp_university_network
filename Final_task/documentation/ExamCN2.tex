\documentclass[a4paper,10pt]{book}
\usepackage[utf8]{inputenc}
\usepackage{graphicx}
\usepackage{hyperref}
\usepackage{float}
\usepackage{tabularx}
\usepackage{pdfpages}

\restylefloat{table}
% Title Page
\title{Exam Report Communication Networks II}
\author{Daniel Ploeger, Nina Piontek}

\begin{document}
\maketitle
\tableofcontents



\chapter{Network Description}
Because of a construction site near Channel 4 which might accidentally damage the cables 
used for internet connection,
a point-to-point radio link should be implemented to ensure network connection.
In the following, the new network configuration as depicted in \ref{fig:network} will be evaluated.
\begin{figure}[!ht]
  \centering
    \includegraphics[width=0.9\textwidth]{graphics-03.png}
    \label{fig:network}
    \caption{University Network}
\end{figure}

\section{General Discription of the Main Components of the Network}
The network core is a set up with 2 Routers, Main Router (M) and Remote Router (R), connected with a Point-to-Point Link. 
Towards the Main Router the Porters Office' Computer and the Computer of the Professor giving the remote Lecture are connected.
The Main Router is the Gateway for the shared Internet Connection of the University.

Towards the Remote Router the Access Point for WLAN and the Security Camera are connected.
The Access Point provides Wireless Lan for Students at Channel 4. The Computer for the Video Lecture at Channel 4 
is connected via WLAN as well.



\section{Network Services and Requirements}

In this set up the Network performance regarding the Video Lecture is evaluated with the following Services in
use:\\
\begin{enumerate}
 \item WLAN for Webbrowsing
 \item FTP Service for Uploading a file to a Server in the Internet (used by one Student) 
 \item Video Lecture streamed in both directions from the Main Campus to Channel 4
 \item Security Camera streaming from Channel 4 to the Porters Office  at the Main Campus
\end{enumerate}

The network should provide a certain Quality of Service for the Video Lecture and the Security Camera.
The Criteria are:\\
\begin{enumerate}
 \item The Video Data should not take more than 100ms to be delivered to the destination in both directions.
 \item There should only be a loss of data at most of 5$ \% $
\end{enumerate}

\chapter{Network Modelling}
\section{General Assumptions for the Model}
Since the overall Goal is to evaluate the network performance regarding the Qos during the Video Lecture, we will assume an assessment period
of around 90 minutes. The WLAN at Channel 4 will in general be used by the Students, which surf the Web during the Video Lecture and the Student 
who uploads a file. Optional the Camera will be turned on or switched off to evaluate the networks behavior with and without the additional
traffic load.
\subsection{HTTP Traffic}
In order to model the student web browsing behaviour in 
a meaningful way, captured browsing statistics have been provided by the computer center in a trace file. 
This chapter shortly explains the statistical behaviour behind the given HTTP 
requests and responses. The response sample traces (download statistcs) are widely 
distributed in the range of less than 500 Bytes up to more than 4 Mbyte 
(see figure XXX top).

Distribution fitting analysis shows that the statistical behaviour if
 partitioned in a smaller number of bins roughly behaves like a negative 
exponential distribution with a mean value of 0.789 Mbyte (see figure XXX center). 
The phrase "roughly" means that while the statistical goodness of fit tests accepts
 this distribution, the actual data has an overweight of small messages compared to
 the expected values (see figure XXX bottom). An exponential distribution with the
 mean of 0.789 Mbyte is used in this report for simulation nonetheless. 
This is a conservative approach and guarantees an upper bound in the distribution:
 if the actual browsing behaviour uses smaller HTTP responses with a slightly higher
 probability, the QoS assumptions from the following chapters are going to be met 
in practice as well.
\subsection{Network Behaviour - General Expectations}
\begin{minipage}[t]{0.5\textwidth}
\begin{tabular}{|l|l|l|l|l|}
\cline{1-4}
 Link & Data Rate  & Downlink  & Uplink  \\ \cline{1-4}
 WLAN & 54Mbit/s  & HTTP, Video Lecture & FTP,Video Lecture \\ \cline{1-4}
 Access Point $\leftrightarrow$ Remote Router & 100Mbit/s  &HTTP, Video Lecture& FTP  \\ \cline{1-4}
 Camera $\leftrightarrow$ Remote Router & 100Mbit/s  & - & Camera\\ \cline{1-4}
 Remote Router$\leftrightarrow$Main Router&12 Mbit/s&HTTP,Video Lecture& FTP, \\ 
 & & & Video Lecture,Camera \\ \cline{1-4}
 Main Router$\leftrightarrow $ Porters Office &Ideal Connection &Camera& - \\ \cline{1-4}
 Main Router$\leftrightarrow $ Professor & -  &Video Lecture&Video Lecture \\ \cline{1-4}
 Main Router$\leftrightarrow $ Internet &100Mbit/s &HTTP& FTP \\ \cline{1-4}
\end{tabular}
\end{minipage}

The point-to-point radio link between remote and main router has a data rate 
of 12 Mbit/s and by this is the weakest single link within the proposed network. 
All other connection speeds are multiples of this data rate. Since all data has 
to pass this link, for this theoretical behaviour analysis it is considered to 
be the network bottleneck.

The network contains several applications which have their main impact either
upload connection in the other direction. The respective opposite direction of an
 application may be neglected, since it mainly consists of very small messages for connection management or empty network packets. For 
 this reason, theoretical down- and upload utilization is examined independently from each other in this chapter. 

The only exception is the video conference, which utilizes both, up- and download 
equally. The video stream consumes 280 kbit/s for transmission of 1388 Bytes of 
payload plus 12 Byte RTP protocol headers. It does this every 40 ms and in both 
directions.


\chapter{Simulation Results}
\section{Evaluation without Camera}
Since the default configuration for the Links is full-duplex, the performance of the Uplink and the Downlink is evaluated independently.

When the Camera is turned off, the Point-to-Point Link to the Main Campus is shared between the FTP Upload and the Video Lecture (Video Conference).
Since the Video Conference is sending with a constant Data Rate of 280kbit/s, the remaining 
11,72 Mbit/s can be used by the FTP Upload.
\begin{figure}[!ht]
  \centering
    \includegraphics[width=0.9\textwidth]{graphics-02.png}
    \label{fig:g2}
    \caption{Utilization of the Radio Link without the Security Camera streaming}
\end{figure}

On the Downlink the Data Rate of 12Mbit/s is shared between the Students browsing the Web (HTTP Service) and the Video Lecture (Video Conference).
Since the Downlink is utilized by the Video Lecture by 280kbit/s , the remaining 11,7Mbit/s 
can be theoretically used by the HTTP Service.




\subsection{Simulation Results}






\section{Expectations on the Network Behavior - with Camera}


\begin{figure}[!ht]
  \centering
    \includegraphics[width=0.9\textwidth]{graphics-01.png}
    \label{fig:g1}
    \caption{Utilization of the Radio Link with the Security Camera streaming}
\end{figure}











\end{document}          
